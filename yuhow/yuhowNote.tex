\documentclass[a4paper,12pt,dvips]{article}
\usepackage[textwidth=6.5in,textheight=9in]{geometry}
\usepackage[colorlinks=true]{hyperref}
\usepackage{amsmath}
\usepackage{amssymb}
\usepackage{amsthm}
\usepackage[monochrome]{color}
\usepackage{graphicx}     % From LaTeX distribution
%\usepackage{subfigure}    % From CTAN/macros/latex/contrib/supported/subfigure
\usepackage{pst-all}      % From PSTricks
\usepackage{pst-poly}     % From pstricks/contrib/pst-poly
\usepackage{multido}      % From PSTricks
\usepackage[center,footnotesize]{caption}
\usepackage[subrefformat=parens]{subcaption}

\graphicspath{{eps/}}

%\numberwithin{equation}{section}

\newcommand*\diff{\mathop{}\!\mathrm{d}}
\newcommand*\Diff[1]{\mathop{}\!\mathrm{d^#1}}
\newcommand*\defeq{\buildrel{\text{def}}\over{=}}

\begin{document}

\title{Yuhow's Note}
\author{You-Hao Chang}
\date{2015.12.1}

\maketitle

\tableofcontents
%\listoffigures

\hspace{.5cm}

This note contains some questions Yuhow encounter in CESE paper~\cite{CESE_Shin_Chung_Chang_1995}

\section{Questions in section of the $\alpha-\mu$ scheme}
 \begin{enumerate}
  \item As a result of Eq. (2.4), this implies Eq. (2.5).  Why?
  \item Why can Eq. (2.9) be proved using the fact that the total flux of $h^{*}$ leaving the boundary of any space-time region that is the union of any combination of CEs vanishes? Can't figure it out.
  \item What's the finite-difference appromixation?
  \item What's the Lax scheme?
  \item What's the amplification factors? Also what's their meaning/usage in the Leapfrog/DuFort-Frankel scheme?
  \item What's the meaning of "two-level" and "three-level" scheme?
 \end{enumerate}


\begin{thebibliography}{99}
\bibitem{CESE_Shin_Chung_Chang_1995} Sin-Chung Chang, The Method of Space-Time Conservation Element and Solution Element—A New Approach for Solving the Navier-Stokes and Euler Equations. Journal of Computational Physics, 119, 295-324, 1995.
\end{thebibliography}

\end{document}
