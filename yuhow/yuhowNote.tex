\documentclass[a4paper,12pt,dvips]{article}
\usepackage[textwidth=6.5in,textheight=9in]{geometry}
\usepackage[colorlinks=true]{hyperref}
\usepackage{amsmath}
\usepackage{amssymb}
\usepackage{amsthm}
\usepackage[monochrome]{color}
\usepackage{graphicx}     % From LaTeX distribution
%\usepackage{subfigure}    % From CTAN/macros/latex/contrib/supported/subfigure
\usepackage{pst-all}      % From PSTricks
\usepackage{pst-poly}     % From pstricks/contrib/pst-poly
\usepackage{multido}      % From PSTricks
\usepackage[center,footnotesize]{caption}
\usepackage[subrefformat=parens]{subcaption}

\graphicspath{{eps/}}

%\numberwithin{equation}{section}

\newcommand*\diff{\mathop{}\!\mathrm{d}}
\newcommand*\Diff[1]{\mathop{}\!\mathrm{d^#1}}
\newcommand*\defeq{\buildrel{\text{def}}\over{=}}

\begin{document}

\title{Working Note}
\author{You-Hao Chang}
\date{2015.12.1}

\maketitle

\tableofcontents
%\listoffigures

\hspace{.5cm}

This note contains some questions Yuhow encounter in CESE paper~\cite{CESE_Shin_Chung_Chang_1995}

\section{Questions in section of the $\alpha-\mu$ scheme}
 \begin{enumerate}
  \item Why does Eq. (2.4) imply Eq. (2.5). ?
  \item Why can Eq. (2.9) be proved using the fact that the total flux of $h^{*}$ leaving the boundary of any space-time region that is the union of any combination of CEs vanishes? Can't figure it out.
  \item What's the finite-difference appromixation?
  \item What's the meaning of "the $\alpha-\mu$ scheme uses a mesh that is staggered in time"?
  \item What's the Lax scheme?
  \item What's the amplification factors? Also what's their meaning/usage in the Leapfrog/DuFort-Frankel scheme?
  \item What's the meaning of "two-level" and "three-level" scheme?
  \item Why does not solutions of Eq. (2.22) dissipate with time? Or why is "no dissipation" equivalent to "neutrally stable"?
  \item Why the total flux leaving any conservation element is zero? This question is relevant to the definition of Eq. (2.28).
  \item Why does the term "$-\mu\partial^{2}u^{*}(x, t; j, n)/\partial x^{2} $ vanish "in Eq. (2.29)?
  \item (In page 303) Why is the condition that Eq. (2.30) being valid uniformly within an SE stronger than Eq. (2.28) for a higher-order expansion? Are they the same thing but just in different form?
  \item Eq. (2.33) maybe is wrong. The partial derivative in the left-hand side of Eq. (2.33) should be with respect to x instead of t. Will try to double check with author though email. (STATUS: waiting for Prof. Chang's reply)
 \end{enumerate}

\section{Something about Python in Solvcon}
Some libraries which are maybe used in implementation of CESE method: 
 \begin{enumerate}
  \item Numpy, matplotlib
 \end{enumerate}
Useful tools:
 \begin{enumerate}
  \item PyCharm
  \item iPython
 \end{enumerate}

\section{Progress report}
 \begin{enumerate}
  \item ( - 20151211) reading the CESE paper~\cite{CESE_Shin_Chung_Chang_1995}, Achievement: Section 2 and 3.
  \item (20151212 - ) start to focus on Euler solver and make my own 1-D solver which is refer to Tai-Hsiang's code and appendix B of the CESE paper~\cite{CESE_Shin_Chung_Chang_1995}
 \end{enumerate}



\begin{thebibliography}{99}
\bibitem{CESE_Shin_Chung_Chang_1995} Sin-Chung Chang, The Method of Space-Time Conservation Element and Solution Element—A New Approach for Solving the Navier-Stokes and Euler Equations. Journal of Computational Physics, 119, 295-324, 1995.
\end{thebibliography}

\end{document}
