\documentclass[a4paper,12pt,dvips]{article}
\usepackage[textwidth=6.5in,textheight=9in]{geometry}
\usepackage[colorlinks=true]{hyperref}
\usepackage{amsmath}
\usepackage{amssymb}
\usepackage{amsthm}
\usepackage[monochrome]{color}
\usepackage{graphicx}     % From LaTeX distribution
%\usepackage{subfigure}    % From CTAN/macros/latex/contrib/supported/subfigure
\usepackage{pst-all}      % From PSTricks
\usepackage{pst-poly}     % From pstricks/contrib/pst-poly
\usepackage{multido}      % From PSTricks
\usepackage[center,footnotesize]{caption}
\usepackage[subrefformat=parens]{subcaption}

\graphicspath{{eps/}{result_simple/}}

%\numberwithin{equation}{section}

\newcommand*\diff{\mathop{}\!\mathrm{d}}
\newcommand*\Diff[1]{\mathop{}\!\mathrm{d^#1}}
\newcommand*\defeq{\buildrel{\text{def}}\over{=}}

\begin{document}

\title{Introduction to the CESE Method}
\author{Yung-Yu Chen}
\date{2015.10.4}

\maketitle

\tableofcontents
%\listoffigures

\hspace{.5cm}

\section{Space-Time Conservation}

\begin{figure}[htbp]
\centering
\includegraphics{staggered_stencil.eps}
\caption{The space-time staggered stencil.  The dashed cross enclosure denotes
the solution elements (SEs).  The solid dots are the solution points of the
associated solution elements.  $A$, $B$, $C$, $D$, $E$, and $F$ are
intersections of half grid lines.  The rectangle enclosed by the line segments
$\overline{AB}$, $\overline{BC}$, $\overline{CD}$, $\overline{DE}$,
$\overline{EF}$, and $\overline{FA}$ defines a conservation element that allows
the solution to march in time.  The dotted arrow lines show the dependency of
the solution values.}
\label{f:staggered_stencil}
\end{figure}

The Conservation Element and Solution Element (CESE) method\cite{b:chang95}
solves conservation laws by using a non-conventional formulation in space-time.
It significantly differs from well established methods because of two traits:
%
\begin{enumerate}
%
\item It uses an untraditional formulation that unifies space and time.
%
\item It selects both the solution variable and its gradient as unknowns.
%
\end{enumerate}

We start with a one-dimensional hyperbolic partial differential equation (PDE)
\begin{align}
  \frac{\partial u}{\partial t} + \frac{\partial f}{\partial x} = 0
  \label{e:claw1d}
\end{align}
where $u(x, t)$ is the solution variable, and $f(u) = f(x, t)$ is the flux
function.  Let $t = c_0t_0$ have the same dimension as $x$, where $t_0$ is time
and $c_0$ is a scaling constant having the dimension of velocity.  Now $x$ and
$t$ both have the dimension of space.  Rewrite Eq.~(\ref{e:claw1d}) in an
Euclidean space $\mathbb{E}_2$ where $x_1 = x$ and $x_2 = t$
\begin{align*}
  \nabla \cdot \mathbf{h} = 0
\end{align*}
where
\begin{align*}
  \mathbf{h} \defeq \left(\begin{array}{c}
    f \\ u
  \end{array}\right)
\end{align*}
is defined as the space-time flux function.  Aided by the convergence theorem,
it is converted to an integral equation 
\begin{align}
  \int_{S(V)} \mathbf{h} \cdot \diff \boldsymbol\sigma = 0
  \label{e:claw1dint}
\end{align}
where $\diff\boldsymbol\sigma = (\diff x, \diff t)^t$, $V$ the control
volume, and $S(V)$ the control surface.

The CESE method uses a staggered grid, as shown in
Fig.~\ref{f:staggered_stencil}, to develop time-marching schemes.  The
space-time domain is discretized as non-overlapping conservation elements
(CEs), e.g., the rectangle $\square ABCD$ or $\square FADE$.  In the grid,
\begin{align*}
  \Delta x \defeq x_j - x_{j-1}, \quad \Delta t \defeq t^n - t^{n-1}
\end{align*}
Define solution points at the intersection of integer grid lines $t = \ldots,
t^{n-1}, t^n, t^{n+1}, \ldots$ and $x = \ldots, x_{j-1}, x_j, x_{j+1}, \ldots$,
and that of half grid lines $t = \ldots, t^{n-\frac{1}{2}}, t^{n+\frac{1}{2}},
\ldots$ and $x = \ldots, x_{j-\frac{1}{2}}, x_{j+\frac{1}{2}}, \ldots$, which
are marked with the solid dots in Fig.~\ref{f:staggered_stencil}.  Each
solution point is associated with a solution element (SE).

At a solution point, the unknowns include both the solution variable itself
($u_j^n$) and its gradient ($(u_x)_j^n$).  Let $\mathrm{SE}(j,n)$ denote the
solution element having the solution point at $(x_j, t^n)$.  In that solution
element, we approximate the solution variable as
\begin{align}
  &u^*(x,t; j,n) = u_j^n + (x-x_j) (u_x)_j^n + (t-t^n) (u_t)_j^n,
  \label{e:solapprox} \\
  &\quad\quad
  x_{j-\frac{1}{2}} < x < x_{j+\frac{1}{2}}, \quad
  t^{n-\frac{1}{2}} < t < t^{n+\frac{1}{2}} \notag
\end{align}
and
\begin{align*}
  u_j^n \defeq u(x_j, t_n), \quad
  (u_x)_j^n \defeq \left.\frac{\partial u}{\partial x}\right|_{x=x_j,t=t^n},
  \quad
  (u_t)_j^n \defeq \left.\frac{\partial u}{\partial t}\right|_{x=x_j,t=t^n}
  = -(f_u)_j^n (u_x)_j^n
\end{align*}
are treated as constants.  The flux function is approximated as
\begin{align}
  &f^*(x,t; j,n) = f_j^n + (x-x_j) (f_x)_j^n + (t-t^n) (f_t)_j^n,
  \label{e:fluxapprox} \\
  &\quad\quad
  x_{j-\frac{1}{2}} < x < x_{j+\frac{1}{2}}, \quad
  t^{n-\frac{1}{2}} < t < t^{n+\frac{1}{2}} \notag
\end{align}
and
\begin{align*}
  f_j^n \defeq f(x_j, t_n), \quad
  (f_x)_j^n &\defeq \left.\frac{\partial f}{\partial x}\right|_{x=x_j,t=t^n}
  = (f_u)_j^n (u_x)_j^n, \\
  (f_t)_j^n &\defeq \left.\frac{\partial f}{\partial t}\right|_{x=x_j,t=t^n}
  = -(f_u)_j^n (f_u)_j^n (u_x)_j^n
\end{align*}
are treated as constants.

A conservation element is called a basic conservation element (BCE) if it
relates a single solution point at past to a solution point at present.  If two
adjacent BCEs relate two solution points at past to a single solution point at
present, they can be combined to a compound conservation element (CCE).  Figure
\ref{f:conservation_element} shows the relation specifically for
$\mathrm{CE}(j,n)$.  $\mathrm{CE}(j,n)$ is a compound conservation element,
which consists of two basic conservation elements $\mathrm{CE}_-(j,n)$ and
$\mathrm{CE}_+(j,n)$.

\begin{figure}[htbp]
\centering
%
\begin{subfigure}{0.45\textwidth}
\centering
\includegraphics{left_bce.eps}
\caption{The basic conservation element $\mathrm{CE}_-(j,n)$, which is the left
half of the $\mathrm{CE}(j,n)$ (shown in \subref{f:cce}), enclosed by line
segments $\overline{AB}$, $\overline{BC}$, $\overline{CD}$, and
$\overline{DA}$.}
\label{f:left_bce}
\end{subfigure}
\hskip 0.02\textwidth
%
\begin{subfigure}{0.45\textwidth}
\centering
\includegraphics{right_bce.eps}
\caption{The basic conservation element $\mathrm{CE}_+(j,n)$, which is the
right half of $\mathrm{CE}(j,n)$ (shown in \subref{f:cce}), enclosed by line
segments $\overline{AD}$, $\overline{DE}$, $\overline{EF}$, and
$\overline{FA}$.}
\label{f:right_bce}
\end{subfigure}
\\
%
\begin{subfigure}{0.45\textwidth}
\centering
\includegraphics{conservation_element.eps}
\caption{The compound conservation element $\mathrm{CE}(j,n)$, enclosed by line
segments $\overline{AB}$, $\overline{BC}$, $\overline{CD}$, $\overline{DE}$,
$\overline{EF}$, and $\overline{FA}$, as shown in
Fig.~\ref{f:staggered_stencil}.}
\label{f:cce}
\end{subfigure}
%
\caption{Conservation elements.  \subref{f:left_bce} is the basic conservation
element taking the left half of a compound conservation element.
\subref{f:right_bce} is the basic conservation element taking the right half of
a compound conservation element.  \subref{f:cce} is the compound conservation
element.}
\label{f:conservation_element}
\end{figure}

Conservation elements and solution elements allow us to approximate the
original integral equation, Eq.~(\ref{e:claw1dint}).  Define
\begin{align*}
  \mathbf{h}^* \defeq \left(\begin{array}{c}
    f^* \\ u^*
  \end{array}\right)
\end{align*}
Aided by Eqs.~(\ref{e:solapprox}) and (\ref{e:fluxapprox}), we obtain
\begin{align}
  \int_{S(\mathrm{CE})} \mathbf{h}^*\cdot\diff\boldsymbol{\sigma} = 0
  \label{e:claw1dintapprox}
\end{align}
We require Eq.~(\ref{e:claw1dintapprox}) to apply to any conservation element.
Therefore,
\begin{align*}
  \int_{S(\mathrm{CE}(j,n))} \mathbf{h}^*\cdot\diff\boldsymbol{\sigma} = 0,
  \quad
  \int_{S(\mathrm{CE}_-(j,n))} \mathbf{h}^*\cdot\diff\boldsymbol{\sigma} = 0,
  \quad
  \int_{S(\mathrm{CE}_+(j,n))} \mathbf{h}^*\cdot\diff\boldsymbol{\sigma} = 0
\end{align*}
The approximated integral equation will be used to construct time-marching
schemes.

\section{Viscous Model Equation}

\section{Inviscid model equation}

\subsection{Formulation for artificial viscosity}

\subsection{Simplified scheme}

\begin{thebibliography}{99}
\bibitem{b:chang95} Chang, S.-C. (1995), 
``The Method of Space-Time Conservation Element and Solution Element -- 
A New Approach for Solving the Navier-Stokes and Euler Equation.''
Journal of Computational Physics 119(2): 295-324.
\end{thebibliography}

\end{document}
