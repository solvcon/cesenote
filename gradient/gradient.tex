\documentclass[a4paper,12pt,dvips]{article}
\usepackage[textwidth=6.5in,textheight=9in]{geometry}
\usepackage[colorlinks=true]{hyperref}
\usepackage{amsmath}
\usepackage{amssymb}
\usepackage{amsthm}
\usepackage[monochrome]{color}
\usepackage{graphicx}     % From LaTeX distribution
%\usepackage{subfigure}    % From CTAN/macros/latex/contrib/supported/subfigure
\usepackage{pst-all}      % From PSTricks
\usepackage{pst-poly}     % From pstricks/contrib/pst-poly
\usepackage{multido}      % From PSTricks
\usepackage[center,footnotesize]{caption}
\usepackage[subrefformat=parens]{subcaption}

\graphicspath{{eps/}}

%\numberwithin{equation}{section}

\newcommand*\diff{\mathop{}\!\mathrm{d}}
\newcommand*\Diff[1]{\mathop{}\!\mathrm{d^#1}}
\newcommand*\defeq{\buildrel{\text{def}}\over{=}}

\begin{document}

\title{Gradient Approximation}
\author{Yung-Yu Chen}
\date{2016.10.22}

\maketitle

%\tableofcontents
%\listoffigures

The space-time conservation element and solution element (CESE) method supports
unstructured meshes in multi-dimensional space.  The triangular elements are
discussed here.  In Fig.~\ref{f:tri_mesh} there are 6 triangles shown.

\begin{figure}[htbp]
\centering
\includegraphics{tri_mesh.eps}
\caption{Triangular mesh in two-dimensional space.}
\label{f:tri_mesh}
\end{figure}

The CESE method place the solutions at the geometrical centers of the elements.
The element centers alone with the mesh vertices consist of the conservation
elements.  See Fig.~\ref{f:ce}.

\begin{figure}[htbp]
\centering
\includegraphics{ce.eps}
\caption{Conservation elements of triangular meshes.}
\label{f:ce}
\end{figure}

Consider a function $u(\mathbf{x})$ in $\mathbb{E}^N$ space.  Say on $N+1$
non-colinear points $\mathbf{x}^{(0)}, \mathbf{x}^{(1)}, \ldots,
\mathbf{x}^{(N)}$, the value of the function is known as $u^{(\mu)} \defeq
u(\mathbf{x}^{(\mu)}), \mu = 0, 1, \ldots, N$.  The gradient of $u$ at
$\mathbf{x}^{(0)}$ can be approximated by
\begin{align}
\nabla u(\mathbf{x}^{(0)}) = \mathrm{S}^{-1} \left(\begin{array}{c}
  u^{(1)} - u^{(0)} \\ \vdots \\ u^{(N)} - u^{(0)}
\end{array}\right)
\end{align}
where
\begin{align}
\mathrm{S} \defeq \left(\begin{array}{ccc}
  x^{(1)}_1 - x^{(0)}_1 &
  \cdots &
  x^{(1)}_N - x^{(0)}_N \\
  \vdots & \ddots & \vdots \\
  x^{(N)}_1 - x^{(0)}_1 &
  \cdots &
  x^{(N)}_N - x^{(0)}_N
\end{array}\right)
\end{align}

%\addcontentsline{toc}{section}{References}
%\begin{thebibliography}{99}
%\bibitem{laney_computational_1998} C. B. Laney, Computational Gasdynamics.
%Cambridge: Cambridge University Press, 1998.
%\end{thebibliography}

\end{document}
