\documentclass[a4paper,12pt,dvips]{article}
\usepackage[textwidth=6.5in,textheight=9in]{geometry}
\usepackage[colorlinks=true]{hyperref}
\usepackage{amsmath}
\usepackage{amssymb}
\usepackage{amsthm}
\usepackage[monochrome]{color}
\usepackage{graphicx}     % From LaTeX distribution
%\usepackage{subfigure}    % From CTAN/macros/latex/contrib/supported/subfigure
\usepackage{pst-all}      % From PSTricks
\usepackage{pst-poly}     % From pstricks/contrib/pst-poly
\usepackage{multido}      % From PSTricks
\usepackage[center,footnotesize]{caption}
\usepackage[subrefformat=parens]{subcaption}

\graphicspath{{eps/}}

%\numberwithin{equation}{section}

\newcommand*\diff{\mathop{}\!\mathrm{d}}
\newcommand*\Diff[1]{\mathop{}\!\mathrm{d^#1}}
\newcommand*\defeq{\buildrel{\text{def}}\over{=}}

\begin{document}

\title{Wall Boundary Treatment for Gas Dynamics}
\author{Yung-Yu Chen}
\date{2016.3.10}

\maketitle

\tableofcontents
%\listoffigures

\hspace{.5cm}

Regarding gas dynamics, almost all problem setups involve walls.  For the Euler
equation, which doesn't have a diffusion term, inviscid solid walls are
commonplace.  A reflection flow is usually used to treat the solid-wall
boundary condition\cite{laney_computational_1998}.  The technique is similar to
the method of image for the Laplace equation.  For the Euler equation, if there
is a flow mirroring the flow inside the wall, the boundary will not be
penetrated.  The reflection flow is also called a ghost flow because it is
artificially created outside the computing domain.

\section{Rotational Coordinate Transform}

Consider the general case in the three-dimensional Euclidean coordinate system.
Call $(x_1, x_2, x_3)$ a global coordinate system of which
%
$\hat{\mathbf{x}}_1$, $\hat{\mathbf{x}}_2$, and $\hat{\mathbf{x}}_3$
%
are the unit vectors along the axes, respectively.

\begin{figure}[htbp]
\centering
\includegraphics{boundary_coordinate.eps}
\caption{The local coordinate system of a boundary face.}
\label{f:boundary_coordinate}
\end{figure}

Of a certain boundary face, let
%
\begin{align}
  \hat{\boldsymbol{\xi}}_1
  = \xi_{11} \hat{\mathbf{x}}_1
  + \xi_{12} \hat{\mathbf{x}}_2
  + \xi_{13} \hat{\mathbf{x}}_3
  \label{e:coord:xi1}
\end{align}
%
be the unit normal vector.  See Fig.~\ref{f:boundary_coordinate}.  By choosing
two other orthogonal unit vectors
%
\begin{align}
\begin{aligned}
  \hat{\boldsymbol{\xi}}_2
 &= \xi_{21} \hat{\mathbf{x}}_1
  + \xi_{22} \hat{\mathbf{x}}_2
  + \xi_{23} \hat{\mathbf{x}}_3
  \\
  \hat{\boldsymbol{\xi}}_3
 &= \xi_{31} \hat{\mathbf{x}}_1
  + \xi_{32} \hat{\mathbf{x}}_2
  + \xi_{33} \hat{\mathbf{x}}_3
\end{aligned}
\label{e:coord:xi23}
\end{align}
%
that fulfill the right-hand rule:
%
$\hat{\boldsymbol{\xi}}_3 = \hat{\boldsymbol{\xi}}_1 \times
\hat{\boldsymbol{\xi}}_2$,
%
$\hat{\boldsymbol{\xi}}_1 = \hat{\boldsymbol{\xi}}_2 \times
\hat{\boldsymbol{\xi}}_3$, and
%
$\hat{\boldsymbol{\xi}}_2 = \hat{\boldsymbol{\xi}}_3 \times
\hat{\boldsymbol{\xi}}_1$,
%
a local coordinate system $(\xi_1, \xi_2, \xi_3)$ is defined on the boundary
face.  Note, the two coordinate systems are assumed to have the same origin, so
that they can be transformed to each other by pure rotation.  (They are drawn
at different locations in Fig.~\ref{f:boundary_coordinate} to make the
schematics clean.)

\subsection{Vector Rotation}

A certain vector, or point, in space can be represented using both of the
coordinate systems.  That is, with $\mathbf{p}$ denoting the point in $(x_1,
x_2, x_3)$, and $\bar{\mathbf{p}}$ denoting it in $(\xi_1, \xi_2, \xi_3)$, the
following equation holds
\begin{align*}
  \mathbf{p}
  = p_1 \hat{\mathbf{x}}_1 + p_2 \hat{\mathbf{x}}_2 + p_3 \hat{\mathbf{x}}_3
  = \bar{\mathbf{p}}
  = \bar{p}_1 \hat{\boldsymbol{\xi}}_1
  + \bar{p}_2 \hat{\boldsymbol{\xi}}_2
  + \bar{p}_3 \hat{\boldsymbol{\xi}}_3
\end{align*}
Substitute Eqs.~(\ref{e:coord:xi1}) and (\ref{e:coord:xi23}) into the above
equation:
\begin{align*}
  p_1 \hat{\mathbf{x}}_1 + p_2 \hat{\mathbf{x}}_2 + p_3 \hat{\mathbf{x}}_3
  = \; 
   &\bar{p}_1 (\xi_{11} \hat{\mathbf{x}}_1 + \xi_{12} \hat{\mathbf{x}}_2
             + \xi_{13} \hat{\mathbf{x}}_3)
  \\
  + \;
   &\bar{p}_2 (\xi_{21} \hat{\mathbf{x}}_1 + \xi_{22} \hat{\mathbf{x}}_2
             + \xi_{23} \hat{\mathbf{x}}_3)
  \\
  + \;
   &\bar{p}_3 (\xi_{31} \hat{\mathbf{x}}_1 + \xi_{32} \hat{\mathbf{x}}_2
             + \xi_{33} \hat{\mathbf{x}}_3)
\end{align*}
Because $\hat{\mathbf{x}}_1$, $\hat{\mathbf{x}}_2$, and $\hat{\mathbf{x}}_3$
are orthogonal, reorganizing the equation gives the transformation relation
between the two coordinate systems $(x_1, x_2, x_3)$, and $(\xi_1, \xi_2,
\xi_3)$ as three independent equations
\begin{align}
\begin{aligned}
  p_1 &= \bar{p}_1 \xi_{11} + \bar{p}_2 \xi_{21} + \bar{p}_3 \xi_{31}
  \\
  p_2 &= \bar{p}_1 \xi_{12} + \bar{p}_2 \xi_{22} + \bar{p}_3 \xi_{32}
  \\
  p_3 &= \bar{p}_1 \xi_{13} + \bar{p}_2 \xi_{23} + \bar{p}_3 \xi_{33}
\end{aligned}
\label{e:rotrel}
\end{align}

If the vector $\mathbf{p}$ (and $\bar{\mathbf{p}}$) is written as a column
vector
\begin{align*}
  \mathbf{p} = \left(\begin{array}{c}
    p_1 \\ p_2 \\ p_3
  \end{array}\right), \;
  \bar{\mathbf{p}} = \left(\begin{array}{c}
    \bar{p}_1 \\ \bar{p}_2 \\ \bar{p}_3
  \end{array}\right)
\end{align*}
The transformation relation Eq.~(\ref{e:rotrel}) can be rewritten in the
matrix-vector form
\begin{align}
  \mathbf{p} = \mathrm{R}\bar{\mathbf{p}},
  \; \mbox{and conversely} \;
  \bar{\mathbf{p}} = \mathrm{R}^{-1}\mathbf{p}
  \label{e:rotmv}
\end{align}
where
\begin{align}
  \mathrm{R}
  \defeq \left(\begin{array}{ccc}
    \xi_{11} & \xi_{21} & \xi_{31} \\
    \xi_{12} & \xi_{22} & \xi_{32} \\
    \xi_{13} & \xi_{23} & \xi_{33}
  \end{array}\right)
  = \left(\begin{array}{ccc}
    \hat{\boldsymbol{\xi}}_1 &
    \hat{\boldsymbol{\xi}}_2 &
    \hat{\boldsymbol{\xi}}_3
  \end{array}\right)
  \label{e:rotmat}
\end{align}
Note, if $\mathbf{x}$ is used to represent a position vector in $(x_1, x_2,
x_3)$, and $\boldsymbol{\xi}$ is used to represent a position vector in
$(\xi_1, \xi_2, \xi_3)$, then when the two position vectors point to the same
location, $\mathbf{p} = \mathbf{x}$, $\bar{\mathbf{p}} = \boldsymbol{\xi}$, and
$\mathbf{x} = \boldsymbol{\xi}$.

\subsection{Jacobian Rotation}

Consider a vector function $\mathbf{v}(\mathbf{x})$ in Euclidean $n$-space.  A
Jacobian is defined as
\begin{align*}
  \dfrac{\partial\mathbf{v}}{\partial\mathbf{x}}
  = \dfrac{\partial(v_1, \ldots, v_n)}{\partial(x_1, \ldots, x_n)},
  \; \mbox{or} \;
  \renewcommand{\arraystretch}{2.2}
  \dfrac{\partial\mathbf{v}}{\partial\mathbf{x}}
  = \left(\begin{array}{ccc}
    \dfrac{\partial v_1}{\partial x_1} &
    \cdots &
    \dfrac{\partial v_2}{\partial x_n} \\
    \vdots & \ddots & \vdots \\
    \dfrac{\partial v_n}{\partial x_1} &
    \cdots &
    \dfrac{\partial v_n}{\partial x_n}
  \end{array}\right)
\end{align*}
Write the Jacobian matrices of $\mathbf{v}$ in both the global and local
coordinate systems, respectively:
\begin{align*}
  \renewcommand{\arraystretch}{2.2}
  \mathrm{J}
  = \left(\begin{array}{ccc}
    \dfrac{\partial v_1}{\partial x_1} &
    \dfrac{\partial v_1}{\partial x_2} &
    \dfrac{\partial v_1}{\partial x_3} \\
    \dfrac{\partial v_2}{\partial x_1} &
    \dfrac{\partial v_2}{\partial x_2} &
    \dfrac{\partial v_2}{\partial x_3} \\
    \dfrac{\partial v_3}{\partial x_1} &
    \dfrac{\partial v_3}{\partial x_2} &
    \dfrac{\partial v_3}{\partial x_3}
  \end{array}\right)
  \; \mbox{and} \;
  \bar{\mathrm{J}}
  = \left(\begin{array}{ccc}
    \dfrac{\partial \bar{v}_1}{\partial\xi_1} &
    \dfrac{\partial \bar{v}_1}{\partial\xi_2} &
    \dfrac{\partial \bar{v}_1}{\partial\xi_3} \\
    \dfrac{\partial \bar{v}_2}{\partial\xi_1} &
    \dfrac{\partial \bar{v}_2}{\partial\xi_2} &
    \dfrac{\partial \bar{v}_2}{\partial\xi_3} \\
    \dfrac{\partial \bar{v}_3}{\partial\xi_1} &
    \dfrac{\partial \bar{v}_3}{\partial\xi_2} &
    \dfrac{\partial \bar{v}_3}{\partial\xi_3}
  \end{array}\right)
\end{align*}
Aided by $\mathbf{v} = \mathrm{R}\bar{\mathbf{v}}$, write
\begin{align*}
  \frac{\partial v_i}{\partial x_j}
  = \frac{\partial (\mathrm{R}\bar{\mathbf{v}})_i}{\partial x_j}
  = \frac{\partial}{\partial x_j}
    \left( \sum_{k=1}^3 \xi_{ki}\bar{v}_k \right)
  = \sum_{k=1}^3
    \left( \xi_{ki} \frac{\partial \bar{v}_k}{\partial x_j} \right),
  \; i,j = 1, 2, 3
\end{align*}
By following the chain rule, write
\begin{align*}
  \frac{\partial \bar{v}_k}{\partial x_j}
  = \sum_{l=1}^3
    \left(
      \frac{\partial \bar{v}_k}{\partial \xi_l}
      \frac{\partial \xi_l}{\partial x_j}
    \right)
  = \sum_{l=1}^3
    \left( \xi_{lj} \frac{\partial \bar{v}_k}{\partial \xi_l} \right),
  \; j,k = 1, 2, 3
\end{align*}
Combining the above two equations gives
\begin{align*}
  \frac{\partial v_i}{\partial x_j}
  = \sum_{k,l=1}^3
    \left(
      \xi_{ki} \xi_{lj} \frac{\partial \bar{v}_k}{\partial \xi_l}
    \right),
  \; i,j = 1, 2, 3
\end{align*}
The matrix-vector form of the above equations is
\begin{align}
  \mathrm{J} = \mathrm{R\bar{J}R^{-1}},
  \; \mbox{with the inverse transformation} \;
  \bar{\mathrm{J}} = \mathrm{R^{-1}JR}
  \label{e:rotjac}
\end{align}

\section{No-Penetration Boundary Condition}

The no-penetration condition enforces that nothing flows through the boundary.
That is, at all time, the vector component in the $\xi_1$-direction must be
zero at the boundary:
\begin{align*}
  \Bigl.\bar{v}_1\Bigr|_{\xi_1=0} = 0
\end{align*}
For the concept of ``flow'' to make sense, here the vector $\mathbf{v}$ is
considered as velocity.

A common treatment is to use a reflection flow with a ghost cell.  A ghost cell
is mirror image of an interior cell.  By creating a ghost flow mirroring the
interior flow, the no-penetration condition is satisfied.  Let the double prime
(${}''$) denote the value with the ghost cell (outside the computing domain),
and the single prime (${}'$) denote the value with the interior cell.  Write
\begin{align*}
  \bar{v}''_1 = -\bar{v}'_1
\end{align*}
Spatial derivatives are part of solutions in the CESE method, and they need to
be treated as well:
\begin{align*}
  \frac{\partial \bar{v}''_1}{\partial\xi_1} =
  \frac{\partial \bar{v}'_1}{\partial\xi_1}, \quad
  \frac{\partial \bar{v}''_1}{\partial\xi_2} =
  -\frac{\partial \bar{v}'_1}{\partial\xi_2}, \quad
  \frac{\partial \bar{v}''_1}{\partial\xi_3} =
  -\frac{\partial \bar{v}'_1}{\partial\xi_3}
\end{align*}
The tangential components should remain unchanged.  Their treatment is
\begin{align*}
\begin{gathered}
  \bar{v}''_i = \bar{v}'_i, \\
  \frac{\partial \bar{v}''_i}{\partial\xi_1} =
  -\frac{\partial \bar{v}'_i}{\partial\xi_1}, \quad
  \frac{\partial \bar{v}''_i}{\partial\xi_2} =
  \frac{\partial \bar{v}'_i}{\partial\xi_2}, \quad
  \frac{\partial \bar{v}''_i}{\partial\xi_3} =
  \frac{\partial \bar{v}'_i}{\partial\xi_3}
\end{gathered}
\end{align*}
where $i = 2, 3$.

Aided by letting
\begin{align*}
  \mathrm{N} \defeq \left(\begin{array}{ccc}
    1 & 0 & 0 \\ 0 & -1 & 0 \\ 0 & 0 & -1
  \end{array}\right)
\end{align*}
the treatments can be written in the matrix-vector form in the local coordinate
system $(\xi_1, \xi_2, \xi_3)$:
\begin{gather*}
  \left(\begin{array}{c}
    \bar{v}''_1 \\ \bar{v}''_2 \\ \bar{v}''_3
  \end{array}\right)
  = -\mathrm{N}
    \left(\begin{array}{c}
      \bar{v}'_1 \\ \bar{v}'_2 \\ \bar{v}'_3
    \end{array}\right),
  \\
  \renewcommand{\arraystretch}{2.2}
  \left(\begin{array}{ccc}
    \dfrac{\partial \bar{v}''_1}{\partial\xi_1} &
    \dfrac{\partial \bar{v}''_1}{\partial\xi_2} &
    \dfrac{\partial \bar{v}''_1}{\partial\xi_3} \\
    \dfrac{\partial \bar{v}''_2}{\partial\xi_1} &
    \dfrac{\partial \bar{v}''_2}{\partial\xi_2} &
    \dfrac{\partial \bar{v}''_2}{\partial\xi_3} \\
    \dfrac{\partial \bar{v}''_3}{\partial\xi_1} &
    \dfrac{\partial \bar{v}''_3}{\partial\xi_2} &
    \dfrac{\partial \bar{v}''_3}{\partial\xi_3}
  \end{array}\right)
  = \mathrm{N}
  \left(\begin{array}{ccc}
    \dfrac{\partial \bar{v}'_1}{\partial\xi_1} &
    \dfrac{\partial \bar{v}'_1}{\partial\xi_2} &
    \dfrac{\partial \bar{v}'_1}{\partial\xi_3} \\
    \dfrac{\partial \bar{v}'_2}{\partial\xi_1} &
    \dfrac{\partial \bar{v}'_2}{\partial\xi_2} &
    \dfrac{\partial \bar{v}'_2}{\partial\xi_3} \\
    \dfrac{\partial \bar{v}'_3}{\partial\xi_1} &
    \dfrac{\partial \bar{v}'_3}{\partial\xi_2} &
    \dfrac{\partial \bar{v}'_3}{\partial\xi_3}
  \end{array}\right)
  \mathrm{N}
\end{gather*}
or, more concisely,
\begin{align*}
  \bar{\mathbf{v}}'' = -\mathrm{N}\bar{\mathbf{v}}'
  \; \mbox{and} \;
  \bar{\mathrm{J}}'' = \mathrm{N\bar{J}'N}
\end{align*}

\subsection{Treat Directly in the Global Coordinate System}

It is not convenient nor efficient to specify the boundary condition using the
local coordinate system $(\xi_1, \xi_2, \xi_3)$.  We'd like to use the global
coordinate system $(x_1, x_2, x_3)$.  For the vector itself, perform the
coordinate transform:
\begin{gather*}
  \mathrm{R}^{-1}\mathbf{v}''
  = \bar{\mathbf{v}}''
  = -\mathrm{N}\bar{\mathbf{v}}'
  = -\mathrm{N}\mathrm{R}^{-1}\mathbf{v}'
  \\
  \Rightarrow \;
  \mathbf{v}'' = -\mathrm{RNR^{-1}}\mathbf{v}'
\end{gather*}
The above equation can be rewritten as
\begin{align*}
  \mathbf{v}'' = -\mathrm{T}\mathbf{v}'
\end{align*}
where
\begin{align*}
  \mathrm{T}
 &\defeq \mathrm{RNR^{-1}}
  = \left(\begin{array}{ccc}
    \xi_{11} & \xi_{21} & \xi_{31} \\
    \xi_{12} & \xi_{22} & \xi_{32} \\
    \xi_{13} & \xi_{23} & \xi_{33}
  \end{array}\right)
  \left(\begin{array}{ccc}
    1 & 0 & 0 \\
    0 & -1 & 0 \\
    0 & 0 & -1
  \end{array}\right)
  \left(\begin{array}{ccc}
    \xi_{11} & \xi_{12} & \xi_{13} \\
    \xi_{21} & \xi_{22} & \xi_{23} \\
    \xi_{31} & \xi_{32} & \xi_{33}
  \end{array}\right)
  \\
 &= \left(\begin{array}{ccc}
    \xi_{11}^2       - \xi_{21}^2       - \xi_{31}^2       &
    \xi_{11}\xi_{12} - \xi_{21}\xi_{22} - \xi_{31}\xi_{32} &
    \xi_{11}\xi_{13} - \xi_{21}\xi_{23} - \xi_{31}\xi_{33} \\
    \xi_{11}\xi_{12} - \xi_{21}\xi_{22} - \xi_{31}\xi_{32} &
    \xi_{12}^2       - \xi_{22}^2       - \xi_{32}^2       &
    \xi_{12}\xi_{13} - \xi_{22}\xi_{23} - \xi_{32}\xi_{33} \\
    \xi_{11}\xi_{13} - \xi_{21}\xi_{23} - \xi_{31}\xi_{33} &
    \xi_{12}\xi_{13} - \xi_{22}\xi_{23} - \xi_{32}\xi_{33} &
    \xi_{13}^2       - \xi_{23}^2       - \xi_{33}^2
  \end{array}\right)
\end{align*}
Similarly,
\begin{gather*}
  \mathrm{R^{-1}J''R} = \bar{\mathrm{J}}'' = \mathrm{N\bar{J}'N}
  = \mathrm{NR^{-1}J'RN}
  \\
  \Rightarrow \;
  \mathrm{J}'' = \mathrm{RNR^{-1}J'RNR^{-1}}
  \\
  \Rightarrow \;
  \mathrm{J}'' = \mathrm{TJ'T}
\end{gather*}
Spelling out the treatments:
\begin{align}
\begin{aligned}
  \left(\begin{array}{c}
    v''_1 \\ v''_2 \\ v''_3
  \end{array}\right)
  = &\;
 -\left(\begin{array}{ccc}
    T_{11} & T_{12} & T_{13} \\
    T_{21} & T_{22} & T_{23} \\
    T_{31} & T_{32} & T_{33}
  \end{array}\right)
  \left(\begin{array}{c}
    v'_1 \\ v'_2 \\ v'_3
  \end{array}\right),
  \\
  \renewcommand{\arraystretch}{2.2}
  \left(\begin{array}{ccc}
    \dfrac{\partial v''_1}{\partial x_1} &
    \dfrac{\partial v''_1}{\partial x_2} &
    \dfrac{\partial v''_1}{\partial x_3} \\
    \dfrac{\partial v''_2}{\partial x_1} &
    \dfrac{\partial v''_2}{\partial x_2} &
    \dfrac{\partial v''_2}{\partial x_3} \\
    \dfrac{\partial v''_3}{\partial x_1} &
    \dfrac{\partial v''_3}{\partial x_2} &
    \dfrac{\partial v''_3}{\partial x_3}
  \end{array}\right)
  = &\;
  \renewcommand{\arraystretch}{2.2}
  \left(\begin{array}{ccc}
    T_{11} & T_{12} & T_{13} \\
    T_{21} & T_{22} & T_{23} \\
    T_{31} & T_{32} & T_{33}
  \end{array}\right)
  \left(\begin{array}{ccc}
    \dfrac{\partial v'_1}{\partial x_1} &
    \dfrac{\partial v'_1}{\partial x_2} &
    \dfrac{\partial v'_1}{\partial x_3} \\
    \dfrac{\partial v'_2}{\partial x_1} &
    \dfrac{\partial v'_2}{\partial x_2} &
    \dfrac{\partial v'_2}{\partial x_3} \\
    \dfrac{\partial v'_3}{\partial x_1} &
    \dfrac{\partial v'_3}{\partial x_2} &
    \dfrac{\partial v'_3}{\partial x_3}
  \end{array}\right)
  \\ &\;
  \renewcommand{\arraystretch}{2.2}
  \left(\begin{array}{ccc}
    T_{11} & T_{12} & T_{13} \\
    T_{21} & T_{22} & T_{23} \\
    T_{31} & T_{32} & T_{33}
  \end{array}\right)
\end{aligned}
\label{e:treat:vec}
\end{align}
Note $\mathrm{T}$ is a real symmetric matrix.

The treatment for a scalar property ($\alpha$) is the same as that of the
tangential component of $\mathbf{v}$
\begin{align*}
\begin{gathered}
  \alpha'' = \alpha', \\
  \frac{\partial \alpha''}{\partial\xi} =
  -\frac{\partial \alpha'}{\partial\xi}, \quad
  \frac{\partial \alpha''}{\partial\eta} =
  \frac{\partial \alpha'}{\partial\eta}, \quad
  \frac{\partial \alpha''}{\partial\zeta} =
  \frac{\partial \alpha'}{\partial\zeta}
\end{gathered}
\end{align*}
Aided by $\mathrm{R}$, write it in the matrix-vector form
\begin{align}
\begin{gathered}
  \alpha'' = \alpha', \\
  \renewcommand{\arraystretch}{2.2}
  \left(\begin{array}{c}
    \dfrac{\partial\alpha''}{\partial x} \\
    \dfrac{\partial\alpha''}{\partial y} \\
    \dfrac{\partial\alpha''}{\partial z} \\
  \end{array}\right)
  = \mathrm{R}^{-1}
  \left(\begin{array}{c}
    \dfrac{\partial\alpha'}{\partial x} \\
    \dfrac{\partial\alpha'}{\partial y} \\
    \dfrac{\partial\alpha'}{\partial z} \\
  \end{array}\right)
  = \left(\begin{array}{ccc}
    \xi_{11} & \xi_{12} & \xi_{13} \\
    \xi_{21} & \xi_{22} & \xi_{23} \\
    \xi_{31} & \xi_{32} & \xi_{33}
  \end{array}\right)
  \left(\begin{array}{c}
    \dfrac{\partial\alpha'}{\partial x} \\
    \dfrac{\partial\alpha'}{\partial y} \\
    \dfrac{\partial\alpha'}{\partial z} \\
  \end{array}\right)
\end{gathered}
\label{e:treat:sca}
\end{align}

\section{The Euler Equation}

The conservation variables of the Euler equation is defined as
\begin{align*}
  \renewcommand{\arraystretch}{2}
  \mathbf{u} = \left(\begin{array}{c}
    \rho \\ \rho v_1 \\ \rho v_2 \\ \rho v_3 \\
    \rho\left(e + \dfrac{v_1^2+v_2^2+v_3^2}{2}\right)
  \end{array}\right)
  = \left(\begin{array}{c}
    u_1 \\ u_2 \\ u_3 \\ u_4 \\ u_5
  \end{array}\right)
\end{align*}
where $\rho$ is the density, $\mathbf{v}$ the velocity, and $e$ the internal
energy.  Note $\mathbf{u}$ is a 5-component vector while $\mathbf{v}$ is a
3-component Cartesian vector.  $\left(\begin{array}{ccc} u_2 & u_3 & u_4
\end{array}\right)^t$ is a Cartesian vector and $u_1$ and $u_5$ are scalar.
%
Aided by Eq.~(\ref{e:treat:vec}), write
\begin{align}
\begin{aligned}
  \left(\begin{array}{c}
    u''_2 \\ u''_3 \\ u''_4
  \end{array}\right)
  = &\;
 -\left(\begin{array}{ccc}
    T_{11} & T_{12} & T_{13} \\
    T_{21} & T_{22} & T_{23} \\
    T_{31} & T_{32} & T_{33}
  \end{array}\right)
  \left(\begin{array}{c}
    u'_2 \\ u'_3 \\ u'_4
  \end{array}\right),
  \\
  \renewcommand{\arraystretch}{2.2}
  \left(\begin{array}{ccc}
    \dfrac{\partial u''_2}{\partial x_1} &
    \dfrac{\partial u''_2}{\partial x_2} &
    \dfrac{\partial u''_2}{\partial x_3} \\
    \dfrac{\partial u''_3}{\partial x_1} &
    \dfrac{\partial u''_3}{\partial x_2} &
    \dfrac{\partial u''_3}{\partial x_3} \\
    \dfrac{\partial u''_4}{\partial x_1} &
    \dfrac{\partial u''_4}{\partial x_2} &
    \dfrac{\partial u''_4}{\partial x_3}
  \end{array}\right)
  = &\;
  \renewcommand{\arraystretch}{2.2}
  \left(\begin{array}{ccc}
    T_{11} & T_{12} & T_{13} \\
    T_{21} & T_{22} & T_{23} \\
    T_{31} & T_{32} & T_{33}
  \end{array}\right)
  \left(\begin{array}{ccc}
    \dfrac{\partial u'_2}{\partial x_1} &
    \dfrac{\partial u'_2}{\partial x_2} &
    \dfrac{\partial u'_2}{\partial x_3} \\
    \dfrac{\partial u'_3}{\partial x_1} &
    \dfrac{\partial u'_3}{\partial x_2} &
    \dfrac{\partial u'_3}{\partial x_3} \\
    \dfrac{\partial u'_4}{\partial x_1} &
    \dfrac{\partial u'_4}{\partial x_2} &
    \dfrac{\partial u'_4}{\partial x_3}
  \end{array}\right)
  \\ &\;
  \renewcommand{\arraystretch}{2.2}
  \left(\begin{array}{ccc}
    T_{11} & T_{12} & T_{13} \\
    T_{21} & T_{22} & T_{23} \\
    T_{31} & T_{32} & T_{33}
  \end{array}\right)
\end{aligned}
\label{e:treat:euler3d:vec}
\end{align}
%
Aided by Eq.~(\ref{e:treat:sca}), write
\begin{gather}
  u''_1 = u'_1, u''_5 = u'_5,
  \label{e:treat:euler3d:sca}
  \\
  \renewcommand{\arraystretch}{2.2}
  \left(\begin{array}{c}
    \dfrac{\partial u''_1}{\partial x_1} \\
    \dfrac{\partial u''_1}{\partial x_2} \\
    \dfrac{\partial u''_1}{\partial x_3}
  \end{array}\right)
  = \left(\begin{array}{ccc}
    \xi_{11} & \xi_{12} & \xi_{13} \\
    \xi_{21} & \xi_{22} & \xi_{23} \\
    \xi_{31} & \xi_{32} & \xi_{33}
  \end{array}\right)
  \left(\begin{array}{c}
    \dfrac{\partial u'_1}{\partial x_1} \\
    \dfrac{\partial u'_1}{\partial x_2} \\
    \dfrac{\partial u'_1}{\partial x_3}
  \end{array}\right),
  \left(\begin{array}{c}
    \dfrac{\partial u''_5}{\partial x_1} \\
    \dfrac{\partial u''_5}{\partial x_2} \\
    \dfrac{\partial u''_5}{\partial x_3}
  \end{array}\right)
  = \left(\begin{array}{ccc}
    \xi_{11} & \xi_{12} & \xi_{13} \\
    \xi_{21} & \xi_{22} & \xi_{23} \\
    \xi_{31} & \xi_{32} & \xi_{33}
  \end{array}\right)
  \left(\begin{array}{c}
    \dfrac{\partial u'_5}{\partial x_1} \\
    \dfrac{\partial u'_5}{\partial x_2} \\
    \dfrac{\partial u'_5}{\partial x_3}
  \end{array}\right)
  \notag
\end{gather}
%
Equations (\ref{e:treat:euler3d:vec}) and (\ref{e:treat:euler3d:sca}) complete
the treatment for the no-penetration boundary condition of the Euler equation
in the three-dimensional space.

\subsection{Two-Dimensional Treatment}

In two-dimensional space, the coordinate transformation becomes simpler.  See
Fig.~\ref{f:wall_coordinate}.  $\bigtriangleup ABD$ is a cell with the boundary
face $\overline{BD}$ (as a line in the two-dimensional space).  A ghost cell
$\bigtriangleup CBD$ is the mirror image of the interior cell $\bigtriangleup
ABD$.  $\boldsymbol{\xi}_1$ points outward from the interior cell.  There is an
angle $\theta$ between the axes $\xi_1$ and $x_1$, and the rotation matrix is
written as
\begin{align*}
  \mathrm{R} = \left(\begin{array}{cc}
    \cos\theta & -\sin\theta \\
    \sin\theta & \cos\theta
  \end{array}\right)
  = \left(\begin{array}{cc}
    \xi_{11} & \xi_{21} \\
    \xi_{12} & \xi_{22}
  \end{array}\right)
\end{align*}

\begin{figure}[htbp]
\centering
\includegraphics{wall_coordinate.eps}
\caption{Coordinate systems in two-dimensional space.}
\label{f:wall_coordinate}
\end{figure}

The conservation variables of the two-dimensional Euler equation is
\begin{align*}
  \renewcommand{\arraystretch}{2}
  \mathbf{u} = \left(\begin{array}{c}
    \rho \\ \rho v_1 \\ \rho v_2 \\
    \rho\left(e + \dfrac{v_1^2+v_2^2}{2}\right)
  \end{array}\right)
  = \left(\begin{array}{c}
    u_1 \\ u_2 \\ u_3 \\ u_4
  \end{array}\right)
\end{align*}

Aided by Eq.~(\ref{e:treat:vec}), write
\begin{align*}
  \left(\begin{array}{c}
    u''_2 \\ u''_3
  \end{array}\right)
  = &\;
 -\left(\begin{array}{ccc}
    T_{11} & T_{12} \\
    T_{21} & T_{22}
  \end{array}\right)
  \left(\begin{array}{c}
    u'_2 \\ u'_3
  \end{array}\right),
  \\
  \renewcommand{\arraystretch}{2.2}
  \left(\begin{array}{ccc}
    \dfrac{\partial u''_2}{\partial x_1} &
    \dfrac{\partial u''_2}{\partial x_2} \\
    \dfrac{\partial u''_3}{\partial x_1} &
    \dfrac{\partial u''_3}{\partial x_2}
  \end{array}\right)
  = &\;
  \renewcommand{\arraystretch}{2.2}
  \left(\begin{array}{ccc}
    T_{11} & T_{12} \\
    T_{21} & T_{22}
  \end{array}\right)
  \left(\begin{array}{ccc}
    \dfrac{\partial u'_2}{\partial x_1} &
    \dfrac{\partial u'_2}{\partial x_2} \\
    \dfrac{\partial u'_3}{\partial x_1} &
    \dfrac{\partial u'_3}{\partial x_2}
  \end{array}\right)
  \left(\begin{array}{ccc}
    T_{11} & T_{12} \\
    T_{21} & T_{22}
  \end{array}\right)
\end{align*}
%
Aided by Eq.~(\ref{e:treat:sca}), write
\begin{gather*}
  u''_1 = u'_1, u''_4 = u'_4,
  \\
  \renewcommand{\arraystretch}{2.2}
  \left(\begin{array}{c}
    \dfrac{\partial u''_1}{\partial x_1} \\
    \dfrac{\partial u''_1}{\partial x_2}
  \end{array}\right)
  = \left(\begin{array}{ccc}
    \xi_{11} & \xi_{12} \\
    \xi_{21} & \xi_{22}
  \end{array}\right)
  \left(\begin{array}{c}
    \dfrac{\partial u'_1}{\partial x_1} \\
    \dfrac{\partial u'_1}{\partial y_2}
  \end{array}\right),
  \left(\begin{array}{c}
    \dfrac{\partial u''_4}{\partial x_1} \\
    \dfrac{\partial u''_4}{\partial x_2}
  \end{array}\right)
  = \left(\begin{array}{ccc}
    \xi_{11} & \xi_{12} \\
    \xi_{21} & \xi_{22}
  \end{array}\right)
  \left(\begin{array}{c}
    \dfrac{\partial u'_4}{\partial x_1} \\
    \dfrac{\partial u'_4}{\partial x_2}
  \end{array}\right)
\end{gather*}

\begin{thebibliography}{99}
\bibitem{laney_computational_1998} C. B. Laney, Computational Gasdynamics.
Cambridge: Cambridge University Press, 1998.
\bibitem{aris_vectors_1962} Rutherford Aris, Vectors, Tensors, and the Basic
Equations of Fluid Mechanics, Prentice-Hall, 1962.
\bibitem{wang_2d_1999} X.-Y. Wang and S.-C. Chang, ``A 2D Non-Splitting
Unstructured Triangular Mesh Euler Solver Based on the Space-Time Conservation
Element and Solution Element Method,'' Computational Fluid Dynamics JOURNAL,
vol. 8, no. 2, pp. 309–325, 1999.
\end{thebibliography}

\end{document}
