\documentclass[a4paper,12pt,dvips]{article}
\usepackage[textwidth=6.5in,textheight=9in]{geometry}
\usepackage[colorlinks=true]{hyperref}
\usepackage{amsmath}
\usepackage{amssymb}
\usepackage{amsthm}
\usepackage[monochrome]{color}
\usepackage{graphicx}     % From LaTeX distribution
%\usepackage{subfigure}    % From CTAN/macros/latex/contrib/supported/subfigure
\usepackage{pst-all}      % From PSTricks
\usepackage{pst-poly}     % From pstricks/contrib/pst-poly
\usepackage{multido}      % From PSTricks
\usepackage[center,footnotesize]{caption}
\usepackage[subrefformat=parens]{subcaption}

\graphicspath{{eps/}}

%\numberwithin{equation}{section}

\newcommand*\diff{\mathop{}\!\mathrm{d}}
\newcommand*\Diff[1]{\mathop{}\!\mathrm{d^#1}}
\newcommand*\defeq{\buildrel{\text{def}}\over{=}}

\begin{document}

\title{Wall Boundary Condition Treatments for Gas Dynamics}
\author{Yung-Yu Chen}
\date{2015.10.17}

\maketitle

\tableofcontents
%\listoffigures

\hspace{.5cm}

\section{Inviscid Solid Wall}

Solid-wall boundary condition is usually treated with a reflection
flow\cite{laney_computational_1998}.  Similar to the method of image for the
Laplace equation, by creating a flow mirroring the interior flow, the
no-penetration condition is enforced on the boundary for the Euler equation.
Because the reflection flow is artificial and locates outside the computing
domain, it is called the ghost flow.

To implement the treatment, the technique of ghost cell is employed.  Consider
an (interior) mesh element, $\bigtriangleup ABD$, as shown in
Fig.~\ref{f:wall_coordinate}.  $\overline{BD}$ is a boundary line of the
element, below which is outside the computing domain.  A ghost cell
$\bigtriangleup CBD$ is constructed as a mirror image of $\bigtriangleup ABD$
with respect to $\overline{BD}$.

\begin{figure}[htbp]
\centering
\includegraphics{wall_coordinate.eps}
\caption{Coordinate systems of the boundary face.}
\label{f:wall_coordinate}
\end{figure}

For the interior cell $\bigtriangleup ABD$ and the ghost cell $\bigtriangleup
CBD$, a local coordinate system $(\xi, \eta)$ is defined.  The local coordinate
system rotates $\theta$ with respect to the global coordinate system $(x, y)$.
Both coordinate systems are Cartesian.  An arbitrary vector (say, $\mathbf{w}$)
can be transformed from the global coordinate system to the local coordinate
system by using the rotation matrix $\mathrm{R}$
\begin{align*}
\left(\begin{array}{c} w_{\xi} \\ w_{\eta} \end{array}\right)
= \mathrm{R}^{-1}
\left(\begin{array}{c} w_x \\ w_y \end{array}\right), \quad
\mathrm{R} \defeq \left(\begin{array}{cc}
  \cos\theta & -\sin\theta \\ \sin\theta & \cos\theta
\end{array}\right)
\end{align*}
To apply the reflection treatment, vectors should be transformed to the local
coordinate system.

Consider a scalar value $\alpha'$ at point $E'$ in the interior cell
$\bigtriangleup ABD$.  For the reflection ghost flow, the corresponding value
$\alpha''$ at point $E''$ should be exactly the same as $\alpha'$
\begin{align}
\alpha'' = \alpha', \quad
\frac{\partial\alpha''}{\partial\xi} = \frac{\partial\alpha'}{\partial\xi},
\quad
\frac{\partial\alpha''}{\partial\eta} = -\frac{\partial\alpha'}{\partial\eta}
\label{e:reflection_sca}
\end{align}
It can be shown that the gradient of $\alpha''$ or $\alpha'$ in the local
coordinate system is a rotation of that in the global coordinate system
\begin{align}
\renewcommand{\arraystretch}{2.2}
\left(\begin{array}{c}
  \dfrac{\partial\alpha}{\partial\xi} \\ 
  \dfrac{\partial\alpha}{\partial\eta}
\end{array}\right) = \mathrm{R}^{-1}
\left(\begin{array}{c}
  \dfrac{\partial\alpha}{\partial x} \\ 
  \dfrac{\partial\alpha}{\partial y}
\end{array}\right), \quad
\renewcommand{\arraystretch}{2.2}
\left(\begin{array}{c}
  \dfrac{\partial\alpha}{\partial x} \\ 
  \dfrac{\partial\alpha}{\partial y}
\end{array}\right) = \mathrm{R}
\left(\begin{array}{c}
  \dfrac{\partial\alpha}{\partial\xi} \\ 
  \dfrac{\partial\alpha}{\partial\eta}
\end{array}\right) \label{e:reflection_sca_trans}
\end{align}
$\mathrm{R}$ is orthogonal.

Consider a vector (for velocity or momentum)
\begin{align*}
\mathbf{v} = \left(\begin{array}{c}
  v_x \\ v_y
\end{array}\right)_{\text{global}}
= \left(\begin{array}{c}
  v_{\xi} \\ v_{\eta}
\end{array}\right)_{\text{local}}
\end{align*}
The reflection requires
\begin{alignat}{3}
\begin{aligned}
  v''_{\xi}
  &= v'_{\xi}, \quad
  &\frac{\partial v''_{\xi}}{\partial\xi}
  &= \frac{\partial v'_{\xi}}{\partial\xi}, \quad
  &\frac{\partial v''_{\xi}}{\partial\eta}
  &= -\frac{\partial v'_{\xi}}{\partial\eta}, \\
  v''_{\eta}
  &= -v'_{\eta}, \quad
  &\frac{\partial v''_{\eta}}{\partial\xi}
  &= -\frac{\partial v'_{\eta}}{\partial\xi}, \quad
  &\frac{\partial v''_{\eta}}{\partial\eta}
  &= \frac{\partial v'_{\eta}}{\partial\eta}
\end{aligned} \label{e:reflection_vec}
\end{alignat}
It can be shown that the coordinate transform of the vector represented between
the local and global coordinate systems is
\begin{align}
\begin{aligned}
\renewcommand{\arraystretch}{2.2}
\left(\begin{array}{cc}
  \dfrac{\partial v_{\xi}}{\partial\xi} &
  \dfrac{\partial v_{\xi}}{\partial\eta} \\
  \dfrac{\partial v_{\eta}}{\partial\xi} &
  \dfrac{\partial v_{\eta}}{\partial\eta}
\end{array}\right)
= \mathrm{R}^{-1}
\left(\begin{array}{cc}
  \dfrac{\partial v_x}{\partial x} &
  \dfrac{\partial v_x}{\partial y} \\
  \dfrac{\partial v_y}{\partial x} &
  \dfrac{\partial v_y}{\partial y}
\end{array}\right)
\mathrm{R}, \\
\renewcommand{\arraystretch}{2.2}
\left(\begin{array}{cc}
  \dfrac{\partial v_x}{\partial x} &
  \dfrac{\partial v_x}{\partial y} \\
  \dfrac{\partial v_y}{\partial x} &
  \dfrac{\partial v_y}{\partial y}
\end{array}\right)
= \mathrm{R}
\left(\begin{array}{cc}
  \dfrac{\partial v_{\xi}}{\partial\xi} &
  \dfrac{\partial v_{\xi}}{\partial\eta} \\
  \dfrac{\partial v_{\eta}}{\partial\xi} &
  \dfrac{\partial v_{\eta}}{\partial\eta}
\end{array}\right)
\mathrm{R}^{-1}
\end{aligned} \label{e:reflection_vec_trans}
\end{align}

Equations \ref{e:reflection_sca} through \ref{e:reflection_vec_trans} will help
to formulate the treatment of reflection boundary condition for the Euler
equation, of which the conservative variables are
\begin{align*}
\renewcommand{\arraystretch}{2.2}
\mathbf{u} = \left(\begin{array}{c}
  u_1 \\ u_2 \\ u_3 \\ u_4
\end{array}\right)
= \left(\begin{array}{c}
  \rho \\ \rho v_x \\ \rho v_y \\
  \rho\left(e + \dfrac{v_x^2 + v_y^2}{2}\right)
\end{array}\right)
\end{align*}
Let $\mathbf{q}$ be the solution variables in the local coordinate system
\begin{align*}
q_1 = u_1, \quad q_4 = u_4, \quad
\left(\begin{array}{c}
  q_2 \\ q_3
\end{array}\right)
= \mathrm{R}^{-1} \left(\begin{array}{c}
  u_2 \\ u_3
\end{array}\right)
\end{align*}
Let $\mathbf{u}'$ and $\mathbf{q}'$ be the solution variables in the interior
cell, and $\mathbf{u}''$ and $\mathbf{q}''$ in the ghost cell.  The reflection
condition requires
\begin{alignat*}{3}
\begin{aligned}
  q''_1
  &= q'_1, \quad
  &\frac{\partial q''_1}{\partial\xi}
  &= \frac{\partial q'_1}{\partial\xi}, \quad
  &\frac{\partial q''_1}{\partial\eta}
  &= -\frac{\partial q'_1}{\partial\eta}, \\
  q''_2
  &= q'_2, \quad
  &\frac{\partial q''_2}{\partial\xi}
  &= \frac{\partial q'_2}{\partial\xi}, \quad
  &\frac{\partial q''_2}{\partial\eta}
  &= -\frac{\partial q'_2}{\partial\eta}, \\
  q''_3
  &= -q'_3, \quad
  &\frac{\partial q''_3}{\partial\xi}
  &= -\frac{\partial q'_3}{\partial\xi}, \quad
  &\frac{\partial q''_3}{\partial\eta}
  &= \frac{\partial q'_3}{\partial\eta}, \\
  q''_4
  &= q'_4, \quad
  &\frac{\partial q''_4}{\partial\xi}
  &= \frac{\partial q'_4}{\partial\xi}, \quad
  &\frac{\partial q''_4}{\partial\eta}
  &= -\frac{\partial q'_4}{\partial\eta}
\end{aligned}
\end{alignat*}
which can be simplified as
\begin{alignat}{4}
\renewcommand{\arraystretch}{2.2}
\begin{aligned}
  q''_m &= q'_m&,& \quad
  &\left(\begin{array}{c}
    \dfrac{\partial q''_m}{\partial\xi} \\
    \dfrac{\partial q''_m}{\partial\eta}
  \end{array}\right)
  &= \mathrm{N} \left(\begin{array}{c}
    \dfrac{\partial q'_m}{\partial\xi} \\
    \dfrac{\partial q'_m}{\partial\eta}
  \end{array}\right)&,& \quad m = 1, 4, \\
  \left(\begin{array}{c}
    q''_2 \\ q''_3
  \end{array}\right)
  &= \mathrm{N} \left(\begin{array}{c}
    q'_2 \\ q'_3
  \end{array}\right)&,& \quad
  &\left(\begin{array}{cc}
    \dfrac{\partial q''_2}{\partial\xi} &
    \dfrac{\partial q''_2}{\partial\eta} \\
    \dfrac{\partial q''_3}{\partial\xi} &
    \dfrac{\partial q''_3}{\partial\eta}
  \end{array}\right)
  &= \mathrm{N} \left(\begin{array}{cc}
    \dfrac{\partial q'_2}{\partial\xi} &
    \dfrac{\partial q'_2}{\partial\eta} \\
    \dfrac{\partial q'_3}{\partial\xi} &
    \dfrac{\partial q'_3}{\partial\eta}
  \end{array}\right) \mathrm{N}&&
\end{aligned} \label{e:reflection_euler_2d_local}
\end{alignat}
by letting
\begin{align*}
  \mathrm{N} = \left(\begin{array}{cc}
    1 & 0 \\ 0 & -1
  \end{array}\right)
\end{align*}

Equation (\ref{e:reflection_euler_2d_local}) can be written back in the global
coordinate system
\begin{alignat}{4}
\renewcommand{\arraystretch}{2.2}
\begin{aligned}
  u''_m &= u'_m&,& \quad
  &\left(\begin{array}{c}
    \dfrac{\partial u''_m}{\partial x} \\
    \dfrac{\partial u''_m}{\partial y}
  \end{array}\right)
  &= \mathrm{T} \left(\begin{array}{c}
    \dfrac{\partial u'_m}{\partial x} \\
    \dfrac{\partial u'_m}{\partial y}
  \end{array}\right)&,& \quad m = 1, 4, \\
  \left(\begin{array}{c}
    u''_2 \\ u''_3
  \end{array}\right)
  &= \mathrm{T} \left(\begin{array}{c}
    u'_2 \\ u'_3
  \end{array}\right)&,& \quad
  &\left(\begin{array}{cc}
    \dfrac{\partial q''_2}{\partial\xi} &
    \dfrac{\partial q''_2}{\partial\eta} \\
    \dfrac{\partial q''_3}{\partial\xi} &
    \dfrac{\partial q''_3}{\partial\eta}
  \end{array}\right)
  &= \mathrm{T} \left(\begin{array}{cc}
    \dfrac{\partial q'_2}{\partial\xi} &
    \dfrac{\partial q'_2}{\partial\eta} \\
    \dfrac{\partial q'_3}{\partial\xi} &
    \dfrac{\partial q'_3}{\partial\eta}
  \end{array}\right) \mathrm{T}&&
\end{aligned} \label{e:reflection_euler_2d_global}
\end{alignat}
where
\begin{align}
  \mathrm{T} \defeq \mathrm{R}\mathrm{N}\mathrm{R}^{-1}
  = \left(\begin{array}{cc}
    \cos2\theta & \sin2\theta \\ sin2\theta & -\cos2\theta
  \end{array}\right) \label{e:reflection_T}
\end{align}

\begin{thebibliography}{99}
\bibitem{laney_computational_1998} C. B. Laney, Computational Gasdynamics.
Cambridge: Cambridge University Press, 1998.
\end{thebibliography}

\end{document}
