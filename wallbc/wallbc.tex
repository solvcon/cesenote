\documentclass[a4paper,12pt,dvips]{article}
\usepackage[textwidth=6.5in,textheight=9in]{geometry}
\usepackage[colorlinks=true]{hyperref}
\usepackage{amsmath}
\usepackage{amssymb}
\usepackage{amsthm}
\usepackage[monochrome]{color}
\usepackage{graphicx}     % From LaTeX distribution
%\usepackage{subfigure}    % From CTAN/macros/latex/contrib/supported/subfigure
\usepackage{pst-all}      % From PSTricks
\usepackage{pst-poly}     % From pstricks/contrib/pst-poly
\usepackage{multido}      % From PSTricks
\usepackage[center,footnotesize]{caption}
\usepackage[subrefformat=parens]{subcaption}

\graphicspath{{eps/}}

%\numberwithin{equation}{section}

\newcommand*\diff{\mathop{}\!\mathrm{d}}
\newcommand*\Diff[1]{\mathop{}\!\mathrm{d^#1}}
\newcommand*\defeq{\buildrel{\text{def}}\over{=}}

\begin{document}

\title{Inviscid Wall Boundary Treatment for Gas Dynamics}
\author{Yung-Yu Chen}
\date{2015.12.5}

\maketitle

\tableofcontents
%\listoffigures

\hspace{.5cm}

\section{Rotational Coordinate Transform}

Consider the general case in the three-dimensional Euclidean coordinate system.
%
$x_1$, $x_2$, and $x_3$
%
denote the three axes, and
%
$\hat{\mathbf{x}}_1$, $\hat{\mathbf{x}}_2$, and $\hat{\mathbf{x}}_3$
%
the unit vectors along the axes, respectively.  Of a certain boundary face, let
%
\begin{align*}
  \hat{\boldsymbol{\xi}}_1
  = \xi_{11} \hat{\mathbf{x}}_1
  + \xi_{12} \hat{\mathbf{x}}_2
  + \xi_{13} \hat{\mathbf{x}}_3
\end{align*}
%
be the unit normal vector.  See Figure~\ref{f:boundary_coordinate}.  By
choosing two other orthogonal unit vectors
%
\begin{align*}
  \hat{\boldsymbol{\xi}}_2
 &= \xi_{21} \hat{\mathbf{x}}_1
  + \xi_{22} \hat{\mathbf{x}}_2
  + \xi_{23} \hat{\mathbf{x}}_3
  \\
  \hat{\boldsymbol{\xi}}_3
 &= \xi_{31} \hat{\mathbf{x}}_1
  + \xi_{32} \hat{\mathbf{x}}_2
  + \xi_{33} \hat{\mathbf{x}}_3
\end{align*}
%
that fulfill the right-hand rule:
%
$\hat{\boldsymbol{\xi}}_3 = \hat{\boldsymbol{\xi}}_1 \times
\hat{\boldsymbol{\xi}}_2$,
%
$\hat{\boldsymbol{\xi}}_1 = \hat{\boldsymbol{\xi}}_2 \times
\hat{\boldsymbol{\xi}}_3$, and
%
$\hat{\boldsymbol{\xi}}_2 = \hat{\boldsymbol{\xi}}_3 \times
\hat{\boldsymbol{\xi}}_1$,
%
define a local coordinate system $(\xi_1, \xi_2, \xi_3)$ on the boundary face.

\begin{figure}[htbp]
\centering
\includegraphics{boundary_coordinate.eps}
\caption{The local coordinate system of a boundary face.}
\label{f:boundary_coordinate}
\end{figure}

It can be show that a vector
\begin{align*}
  \mathbf{v}
  = \left(\begin{array}{c}
    v_1 \\ v_2 \\ v_3
  \end{array}\right)
\end{align*}
in the global coordinate system $(x_1, x_2, x_3)$ can be transformed to the
local coordinate system $(\xi_1, \xi_2, \xi_3)$ by using the following formula:
\begin{align*}
  \bar{\mathbf{v}}
  = \left(\begin{array}{c}
    \bar{v}_1 \\ \bar{v}_2 \\ \bar{v}_3
  \end{array}\right)
  = \left(\begin{array}{ccc}
    \xi_{11} & \xi_{12} & \xi_{13} \\
    \xi_{21} & \xi_{22} & \xi_{23} \\
    \xi_{31} & \xi_{32} & \xi_{33}
  \end{array}\right)
  \mathbf{v}
\end{align*}
The above equation can be rewritten to
\begin{align*}
  \bar{\mathbf{v}} = \mathrm{R}^{-1}\mathbf{v}
\end{align*}
by introducing
\begin{align*}
  \mathrm{R} \defeq \left(\begin{array}{ccc}
    \boldsymbol{\xi}_1 &
    \boldsymbol{\xi}_2 &
    \boldsymbol{\xi}_3
  \end{array}\right)
  = \left(\begin{array}{ccc}
    \xi_{11} & \xi_{21} & \xi_{31} \\
    \xi_{12} & \xi_{22} & \xi_{32} \\
    \xi_{13} & \xi_{23} & \xi_{33}
  \end{array}\right)
\end{align*}

Consider the spatial derivatives of the vector $\mathbf{v}$.  Write the
Jacobian matrices in the global and local coordinate systems, respectively:
\begin{align*}
  \renewcommand{\arraystretch}{2.2}
  \mathrm{J}
  \defeq \left(\begin{array}{ccc}
    \dfrac{\partial v_1}{\partial x_1} &
    \dfrac{\partial v_1}{\partial x_2} &
    \dfrac{\partial v_1}{\partial x_3} \\
    \dfrac{\partial v_2}{\partial x_1} &
    \dfrac{\partial v_2}{\partial x_2} &
    \dfrac{\partial v_2}{\partial x_3} \\
    \dfrac{\partial v_3}{\partial x_1} &
    \dfrac{\partial v_3}{\partial x_2} &
    \dfrac{\partial v_3}{\partial x_3}
  \end{array}\right)
  \; \mbox{and} \;
  \bar{\mathrm{J}}
  \defeq \left(\begin{array}{ccc}
    \dfrac{\partial \bar{v}_1}{\partial\xi_1} &
    \dfrac{\partial \bar{v}_1}{\partial\xi_2} &
    \dfrac{\partial \bar{v}_1}{\partial\xi_3} \\
    \dfrac{\partial \bar{v}_2}{\partial\xi_1} &
    \dfrac{\partial \bar{v}_2}{\partial\xi_2} &
    \dfrac{\partial \bar{v}_2}{\partial\xi_3} \\
    \dfrac{\partial \bar{v}_3}{\partial\xi_1} &
    \dfrac{\partial \bar{v}_3}{\partial\xi_2} &
    \dfrac{\partial \bar{v}_3}{\partial\xi_3}
  \end{array}\right)
\end{align*}
Aided by $\mathbf{v} = \mathrm{R}\bar{\mathbf{v}}$, write
\begin{align*}
  \frac{\partial v_i}{\partial x_j}
  = \frac{\partial (\mathrm{R}\bar{\mathbf{v}})_i}{\partial x_j}
  = \frac{\partial}{\partial x_j}
    \left( \sum_{k=1}^3 \xi_{ki}\bar{v}_k \right)
  = \sum_{k=1}^3
    \left( \xi_{ki} \frac{\partial \bar{v}_k}{\partial x_j} \right),
  \; i,j = 1, 2, 3
\end{align*}
The chain rule further gives
\begin{align*}
  \frac{\partial \bar{v}_k}{\partial x_j}
  = \sum_{l=1}^3
    \left(
      \frac{\partial \bar{v}_k}{\partial \xi_l}
      \frac{\partial \xi_l}{\partial x_j}
    \right)
  = \sum_{l=1}^3
    \left( \xi_{lj} \frac{\partial \bar{v}_k}{\partial \xi_l} \right),
  \; j,k = 1, 2, 3
\end{align*}
Combine the above two equations and obtain
\begin{align*}
  \frac{\partial v_i}{\partial x_j}
  = \sum_{k,l=1}^3
    \left(
      \xi_{ki} \xi_{lj} \frac{\partial \bar{v}_k}{\partial \xi_l}
    \right),
  \; i,j = 1, 2, 3
\end{align*}
The vector-matrix form of the above transformation is
\begin{align*}
  \mathrm{J} = \mathrm{R\bar{J}R^{-1}}
\end{align*}
And the inverse transformation is
\begin{align*}
  \bar{\mathrm{J}} = \mathrm{R^{-1}JR}
\end{align*}

\section{No-Penetration Boundary Condition}

The no-penetration condition enforces that nothing flows through the boundary.
That is, at all time, the vector component in the $\xi_1$-direction must be
zero on the boundary:
\begin{align*}
  \Bigl.\bar{v}_1\Bigr|_{\xi_1=0} = 0
\end{align*}
For the concept of ``flow'' to make sense, here the vector $\mathbf{v}$ is
considered as velocity.

A common treatment of the no-penetration boundary condition is to use
reflection flow with the ghost cell.  The ghost cell is mirror image of the
interior cell.  By creating a ghost flow mirroring the interior flow, the above
no-penetration condition is satisfied.  Let the double prime ${}''$ denote the
value of the ghost cell (outside the computing domain), and the single prime
${}'$ denote the value of the interior cell, write
\begin{align*}
  \bar{v}''_1 = -\bar{v}'_1
\end{align*}
Spatial derivatives are part of solutions in the CESE method, and they need to
be treated as well:
\begin{align*}
  \frac{\partial \bar{v}''_1}{\partial\xi_1} =
  \frac{\partial \bar{v}'_1}{\partial\xi_1}, \quad
  \frac{\partial \bar{v}''_1}{\partial\xi_2} =
  -\frac{\partial \bar{v}'_1}{\partial\xi_2}, \quad
  \frac{\partial \bar{v}''_1}{\partial\xi_3} =
  -\frac{\partial \bar{v}'_1}{\partial\xi_3}
\end{align*}
The tangential components should remain unchanged.  Their treatment is
\begin{align*}
\begin{gathered}
  \bar{v}''_i = \bar{v}'_i, \\
  \frac{\partial \bar{v}''_i}{\partial\xi_1} =
  -\frac{\partial \bar{v}'_i}{\partial\xi_1}, \quad
  \frac{\partial \bar{v}''_i}{\partial\xi_2} =
  \frac{\partial \bar{v}'_i}{\partial\xi_2}, \quad
  \frac{\partial \bar{v}''_i}{\partial\xi_3} =
  \frac{\partial \bar{v}'_i}{\partial\xi_3}
\end{gathered}
\end{align*}
where $i = 2, 3$.

Aided by letting
\begin{align*}
  \mathrm{N} \defeq \left(\begin{array}{ccc}
    1 & 0 & 0 \\ 0 & -1 & 0 \\ 0 & 0 & -1
  \end{array}\right)
\end{align*}
the treatments can be written in the vector-matrix form in the local coordinate
system $(\xi_1, \xi_2, \xi_3)$:
\begin{gather*}
  \left(\begin{array}{c}
    \bar{v}''_1 \\ \bar{v}''_2 \\ \bar{v}''_3
  \end{array}\right)
  = -\mathrm{N}
    \left(\begin{array}{c}
      \bar{v}'_1 \\ \bar{v}'_2 \\ \bar{v}'_3
    \end{array}\right),
  \\
  \renewcommand{\arraystretch}{2.2}
  \left(\begin{array}{ccc}
    \dfrac{\partial \bar{v}''_1}{\partial\xi_1} &
    \dfrac{\partial \bar{v}''_1}{\partial\xi_2} &
    \dfrac{\partial \bar{v}''_1}{\partial\xi_3} \\
    \dfrac{\partial \bar{v}''_2}{\partial\xi_1} &
    \dfrac{\partial \bar{v}''_2}{\partial\xi_2} &
    \dfrac{\partial \bar{v}''_2}{\partial\xi_3} \\
    \dfrac{\partial \bar{v}''_3}{\partial\xi_1} &
    \dfrac{\partial \bar{v}''_3}{\partial\xi_2} &
    \dfrac{\partial \bar{v}''_3}{\partial\xi_3}
  \end{array}\right)
  = \mathrm{N}
  \left(\begin{array}{ccc}
    \dfrac{\partial \bar{v}'_1}{\partial\xi_1} &
    \dfrac{\partial \bar{v}'_1}{\partial\xi_2} &
    \dfrac{\partial \bar{v}'_1}{\partial\xi_3} \\
    \dfrac{\partial \bar{v}'_2}{\partial\xi_1} &
    \dfrac{\partial \bar{v}'_2}{\partial\xi_2} &
    \dfrac{\partial \bar{v}'_2}{\partial\xi_3} \\
    \dfrac{\partial \bar{v}'_3}{\partial\xi_1} &
    \dfrac{\partial \bar{v}'_3}{\partial\xi_2} &
    \dfrac{\partial \bar{v}'_3}{\partial\xi_3}
  \end{array}\right)
  \mathrm{N}
\end{gather*}
or, more concisely,
\begin{align*}
  \bar{\mathbf{v}}'' = -\mathrm{N}\bar{\mathbf{v}}'
  \; \mbox{and} \;
  \bar{\mathrm{J}}'' = \mathrm{N\bar{J}'N}
\end{align*}

It is not convenient to specify the boundary condition using the local
coordinate system $(\xi_1, \xi_2, \xi_3)$.  We'd like to use the global
coordinate system $(x_1, x_2, x_3)$.  For the vector itself, perform the
coordinate transform:
\begin{gather*}
  \mathrm{R}^{-1}\mathbf{v}''
  = \bar{\mathbf{v}}''
  = -\mathrm{N}\bar{\mathbf{v}}'
  = -\mathrm{N}\mathrm{R}^{-1}\mathbf{v}'
  \\
  \Rightarrow \;
  \mathbf{v}'' = -\mathrm{RNR^{-1}}\mathbf{v}'
\end{gather*}
The above equation can be rewritten as
\begin{align*}
  \mathbf{v}'' = -\mathrm{T}\mathbf{v}'
\end{align*}
by letting
\begin{align*}
  \mathrm{T}
 &\defeq \mathrm{RNR^{-1}}
  = \left(\begin{array}{ccc}
    \xi_{11} & \xi_{21} & \xi_{31} \\
    \xi_{12} & \xi_{22} & \xi_{32} \\
    \xi_{13} & \xi_{23} & \xi_{33}
  \end{array}\right)
  \left(\begin{array}{ccc}
    1 & 0 & 0 \\
    0 & -1 & 0 \\
    0 & 0 & -1
  \end{array}\right)
  \left(\begin{array}{ccc}
    \xi_{11} & \xi_{12} & \xi_{13} \\
    \xi_{21} & \xi_{22} & \xi_{23} \\
    \xi_{31} & \xi_{32} & \xi_{33}
  \end{array}\right)
  \\
 &= \left(\begin{array}{ccc}
    \xi_{11}^2       - \xi_{21}^2       - \xi_{31}^2       &
    \xi_{11}\xi_{12} - \xi_{21}\xi_{22} - \xi_{31}\xi_{32} &
    \xi_{11}\xi_{13} - \xi_{21}\xi_{23} - \xi_{31}\xi_{33} \\
    \xi_{11}\xi_{12} - \xi_{21}\xi_{22} - \xi_{31}\xi_{32} &
    \xi_{12}^2       - \xi_{22}^2       - \xi_{32}^2       &
    \xi_{12}\xi_{13} - \xi_{22}\xi_{23} - \xi_{32}\xi_{33} \\
    \xi_{11}\xi_{13} - \xi_{21}\xi_{23} - \xi_{31}\xi_{33} &
    \xi_{12}\xi_{13} - \xi_{22}\xi_{23} - \xi_{32}\xi_{33} &
    \xi_{13}^2       - \xi_{23}^2       - \xi_{33}^2
  \end{array}\right)
\end{align*}
Similarly,
\begin{gather*}
  \mathrm{R^{-1}J''R} = \bar{\mathrm{J}}'' = \mathrm{N\bar{J}'N}
  = \mathrm{NR^{-1}J'RN}
  \\
  \Rightarrow \;
  \mathrm{J}'' = \mathrm{RNR^{-1}J'RNR^{-1}}
  \\
  \Rightarrow \;
  \mathrm{J}'' = \mathrm{TJ'T}
\end{gather*}

The treatment for a scalar property ($\alpha$) is the same as that of the
tangential component of $\mathbf{v}$
\begin{align*}
\begin{gathered}
  \alpha'' = \alpha', \\
  \frac{\partial \alpha''}{\partial\xi} =
  -\frac{\partial \alpha'}{\partial\xi}, \quad
  \frac{\partial \alpha''}{\partial\eta} =
  \frac{\partial \alpha'}{\partial\eta}, \quad
  \frac{\partial \alpha''}{\partial\zeta} =
  \frac{\partial \alpha'}{\partial\zeta}
\end{gathered}
\end{align*}
Aided by $\mathrm{R}$, write it in the vector-matrix form
\begin{align*}
\begin{gathered}
  \alpha'' = \alpha', \\
  \renewcommand{\arraystretch}{2.2}
  \left(\begin{array}{c}
    \dfrac{\partial\alpha''}{\partial x} \\
    \dfrac{\partial\alpha''}{\partial y} \\
    \dfrac{\partial\alpha''}{\partial z} \\
  \end{array}\right)
  = \mathrm{R}^{-1}
  \left(\begin{array}{c}
    \dfrac{\partial\alpha'}{\partial x} \\
    \dfrac{\partial\alpha'}{\partial y} \\
    \dfrac{\partial\alpha'}{\partial z} \\
  \end{array}\right)
\end{gathered}
\end{align*}

\section{Special Case in Two-Dimensional Space}

Regarding gas dynamics, almost all problem setups involve walls.  For the Euler
equation, which doesn't have a diffusion term, inviscid solid walls are
commonplace.  A reflection flow is usually used to treat the solid-wall
boundary condition\cite{laney_computational_1998}.  The technique is similar to
the method of image for the Laplace equation.  For the Euler equation, if there
is a flow mirroring the flow inside the wall, the boundary will not be
penetrated.  The reflection flow is also called a ghost flow because it is
artificially created outside the computing domain.

To produce the ghost flow by mirror image, the concept of ghost cell is
employed.  Consider an (interior) mesh element, $\bigtriangleup ABD$, as shown
in Fig.~\ref{f:wall_coordinate}.  $\overline{BD}$ is a boundary line of the
element, below which is outside the computing domain.  A ghost cell
$\bigtriangleup CBD$ is constructed as a mirror image of $\bigtriangleup ABD$
with respect to $\overline{BD}$.

\begin{figure}[htbp]
\centering
\includegraphics{wall_coordinate.eps}
\caption{Coordinate systems of the boundary face.}
\label{f:wall_coordinate}
\end{figure}

For the interior cell $\bigtriangleup ABD$ and the ghost cell $\bigtriangleup
CBD$, a local coordinate system $(\xi, \eta)$ is defined.  The local coordinate
system rotates $\theta$ with respect to the global coordinate system $(x, y)$.
Both coordinate systems are Cartesian.  An arbitrary vector (say, $\mathbf{w}$)
can be transformed from the global coordinate system to the local coordinate
system by using the rotation matrix $\mathrm{R}$
\begin{align*}
\left(\begin{array}{c} w_{\xi} \\ w_{\eta} \end{array}\right)
= \mathrm{R}^{-1}
\left(\begin{array}{c} w_x \\ w_y \end{array}\right), \quad
\mathrm{R} \defeq \left(\begin{array}{cc}
  \cos\theta & -\sin\theta \\ \sin\theta & \cos\theta
\end{array}\right)
\end{align*}
To apply the reflection treatment, vectors should be transformed to the local
coordinate system.

Consider a scalar value $\alpha'$ at point $E'$ in the interior cell
$\bigtriangleup ABD$.  For the reflection ghost flow, the corresponding value
$\alpha''$ at point $E''$ should be exactly the same as $\alpha'$
\begin{align}
\alpha'' = \alpha', \quad
\frac{\partial\alpha''}{\partial\xi} = \frac{\partial\alpha'}{\partial\xi},
\quad
\frac{\partial\alpha''}{\partial\eta} = -\frac{\partial\alpha'}{\partial\eta}
\label{e:reflection_sca}
\end{align}
It can be shown that the gradient of $\alpha''$ or $\alpha'$ in the local
coordinate system is a rotation of that in the global coordinate system
\begin{align}
\renewcommand{\arraystretch}{2.2}
\left(\begin{array}{c}
  \dfrac{\partial\alpha}{\partial\xi} \\ 
  \dfrac{\partial\alpha}{\partial\eta}
\end{array}\right) = \mathrm{R}^{-1}
\left(\begin{array}{c}
  \dfrac{\partial\alpha}{\partial x} \\ 
  \dfrac{\partial\alpha}{\partial y}
\end{array}\right), \quad
\renewcommand{\arraystretch}{2.2}
\left(\begin{array}{c}
  \dfrac{\partial\alpha}{\partial x} \\ 
  \dfrac{\partial\alpha}{\partial y}
\end{array}\right) = \mathrm{R}
\left(\begin{array}{c}
  \dfrac{\partial\alpha}{\partial\xi} \\ 
  \dfrac{\partial\alpha}{\partial\eta}
\end{array}\right) \label{e:reflection_sca_trans}
\end{align}
$\mathrm{R}$ is orthogonal.

Consider a vector (for velocity or momentum)
\begin{align*}
\mathbf{v} = \left(\begin{array}{c}
  v_x \\ v_y
\end{array}\right)_{\text{global}}
= \left(\begin{array}{c}
  v_{\xi} \\ v_{\eta}
\end{array}\right)_{\text{local}}
\end{align*}
The reflection requires
\begin{alignat}{3}
\begin{aligned}
  v''_{\xi}
  &= v'_{\xi}, \quad
  &\frac{\partial v''_{\xi}}{\partial\xi}
  &= \frac{\partial v'_{\xi}}{\partial\xi}, \quad
  &\frac{\partial v''_{\xi}}{\partial\eta}
  &= -\frac{\partial v'_{\xi}}{\partial\eta}, \\
  v''_{\eta}
  &= -v'_{\eta}, \quad
  &\frac{\partial v''_{\eta}}{\partial\xi}
  &= -\frac{\partial v'_{\eta}}{\partial\xi}, \quad
  &\frac{\partial v''_{\eta}}{\partial\eta}
  &= \frac{\partial v'_{\eta}}{\partial\eta}
\end{aligned} \label{e:reflection_vec}
\end{alignat}
It can be shown that the coordinate transform of the vector represented between
the local and global coordinate systems is
\begin{align}
\begin{aligned}
\renewcommand{\arraystretch}{2.2}
\left(\begin{array}{cc}
  \dfrac{\partial v_{\xi}}{\partial\xi} &
  \dfrac{\partial v_{\xi}}{\partial\eta} \\
  \dfrac{\partial v_{\eta}}{\partial\xi} &
  \dfrac{\partial v_{\eta}}{\partial\eta}
\end{array}\right)
= \mathrm{R}^{-1}
\left(\begin{array}{cc}
  \dfrac{\partial v_x}{\partial x} &
  \dfrac{\partial v_x}{\partial y} \\
  \dfrac{\partial v_y}{\partial x} &
  \dfrac{\partial v_y}{\partial y}
\end{array}\right)
\mathrm{R}, \\
\renewcommand{\arraystretch}{2.2}
\left(\begin{array}{cc}
  \dfrac{\partial v_x}{\partial x} &
  \dfrac{\partial v_x}{\partial y} \\
  \dfrac{\partial v_y}{\partial x} &
  \dfrac{\partial v_y}{\partial y}
\end{array}\right)
= \mathrm{R}
\left(\begin{array}{cc}
  \dfrac{\partial v_{\xi}}{\partial\xi} &
  \dfrac{\partial v_{\xi}}{\partial\eta} \\
  \dfrac{\partial v_{\eta}}{\partial\xi} &
  \dfrac{\partial v_{\eta}}{\partial\eta}
\end{array}\right)
\mathrm{R}^{-1}
\end{aligned} \label{e:reflection_vec_trans}
\end{align}

Equations \ref{e:reflection_sca} through \ref{e:reflection_vec_trans} will help
to formulate the treatment of reflection boundary condition for the Euler
equation, of which the conservative variables are
\begin{align*}
\renewcommand{\arraystretch}{2.2}
\mathbf{u} = \left(\begin{array}{c}
  u_1 \\ u_2 \\ u_3 \\ u_4
\end{array}\right)
= \left(\begin{array}{c}
  \rho \\ \rho v_x \\ \rho v_y \\
  \rho\left(e + \dfrac{v_x^2 + v_y^2}{2}\right)
\end{array}\right)
\end{align*}
Let $\mathbf{q}$ be the solution variables in the local coordinate system
\begin{align*}
q_1 = u_1, \quad q_4 = u_4, \quad
\left(\begin{array}{c}
  q_2 \\ q_3
\end{array}\right)
= \mathrm{R}^{-1} \left(\begin{array}{c}
  u_2 \\ u_3
\end{array}\right)
\end{align*}
Let $\mathbf{u}'$ and $\mathbf{q}'$ be the solution variables in the interior
cell, and $\mathbf{u}''$ and $\mathbf{q}''$ in the ghost cell.  The reflection
condition requires
\begin{alignat*}{3}
\begin{aligned}
  q''_1
  &= q'_1, \quad
  &\frac{\partial q''_1}{\partial\xi}
  &= \frac{\partial q'_1}{\partial\xi}, \quad
  &\frac{\partial q''_1}{\partial\eta}
  &= -\frac{\partial q'_1}{\partial\eta}, \\
  q''_2
  &= q'_2, \quad
  &\frac{\partial q''_2}{\partial\xi}
  &= \frac{\partial q'_2}{\partial\xi}, \quad
  &\frac{\partial q''_2}{\partial\eta}
  &= -\frac{\partial q'_2}{\partial\eta}, \\
  q''_3
  &= -q'_3, \quad
  &\frac{\partial q''_3}{\partial\xi}
  &= -\frac{\partial q'_3}{\partial\xi}, \quad
  &\frac{\partial q''_3}{\partial\eta}
  &= \frac{\partial q'_3}{\partial\eta}, \\
  q''_4
  &= q'_4, \quad
  &\frac{\partial q''_4}{\partial\xi}
  &= \frac{\partial q'_4}{\partial\xi}, \quad
  &\frac{\partial q''_4}{\partial\eta}
  &= -\frac{\partial q'_4}{\partial\eta}
\end{aligned}
\end{alignat*}
which can be simplified as
\begin{alignat}{4}
\renewcommand{\arraystretch}{2.2}
\begin{aligned}
  q''_m &= q'_m&,& \quad
  &\left(\begin{array}{c}
    \dfrac{\partial q''_m}{\partial\xi} \\
    \dfrac{\partial q''_m}{\partial\eta}
  \end{array}\right)
  &= \mathrm{N} \left(\begin{array}{c}
    \dfrac{\partial q'_m}{\partial\xi} \\
    \dfrac{\partial q'_m}{\partial\eta}
  \end{array}\right)&,& \quad m = 1, 4, \\
  \left(\begin{array}{c}
    q''_2 \\ q''_3
  \end{array}\right)
  &= \mathrm{N} \left(\begin{array}{c}
    q'_2 \\ q'_3
  \end{array}\right)&,& \quad
  &\left(\begin{array}{cc}
    \dfrac{\partial q''_2}{\partial\xi} &
    \dfrac{\partial q''_2}{\partial\eta} \\
    \dfrac{\partial q''_3}{\partial\xi} &
    \dfrac{\partial q''_3}{\partial\eta}
  \end{array}\right)
  &= \mathrm{N} \left(\begin{array}{cc}
    \dfrac{\partial q'_2}{\partial\xi} &
    \dfrac{\partial q'_2}{\partial\eta} \\
    \dfrac{\partial q'_3}{\partial\xi} &
    \dfrac{\partial q'_3}{\partial\eta}
  \end{array}\right) \mathrm{N}&&
\end{aligned} \label{e:reflection_euler_2d_local}
\end{alignat}
by letting
\begin{align*}
  \mathrm{N} = \left(\begin{array}{cc}
    1 & 0 \\ 0 & -1
  \end{array}\right)
\end{align*}

Equation (\ref{e:reflection_euler_2d_local}) can be written back in the global
coordinate system
\begin{alignat}{4}
\renewcommand{\arraystretch}{2.2}
\begin{aligned}
  u''_m &= u'_m&,& \quad
  &\left(\begin{array}{c}
    \dfrac{\partial u''_m}{\partial x} \\
    \dfrac{\partial u''_m}{\partial y}
  \end{array}\right)
  &= \mathrm{T} \left(\begin{array}{c}
    \dfrac{\partial u'_m}{\partial x} \\
    \dfrac{\partial u'_m}{\partial y}
  \end{array}\right)&,& \quad m = 1, 4, \\
  \left(\begin{array}{c}
    u''_2 \\ u''_3
  \end{array}\right)
  &= \mathrm{T} \left(\begin{array}{c}
    u'_2 \\ u'_3
  \end{array}\right)&,& \quad
  &\left(\begin{array}{cc}
    \dfrac{\partial u''_2}{\partial\xi} &
    \dfrac{\partial u''_2}{\partial\eta} \\
    \dfrac{\partial u''_3}{\partial\xi} &
    \dfrac{\partial u''_3}{\partial\eta}
  \end{array}\right)
  &= \mathrm{T} \left(\begin{array}{cc}
    \dfrac{\partial u'_2}{\partial\xi} &
    \dfrac{\partial u'_2}{\partial\eta} \\
    \dfrac{\partial u'_3}{\partial\xi} &
    \dfrac{\partial u'_3}{\partial\eta}
  \end{array}\right) \mathrm{T}&&
\end{aligned} \label{e:reflection_euler_2d_global}
\end{alignat}
where
\begin{align}
  \mathrm{T} \defeq \mathrm{R}\mathrm{N}\mathrm{R}^{-1}
  = \left(\begin{array}{cc}
    \cos2\theta & \sin2\theta \\ sin2\theta & -\cos2\theta
  \end{array}\right) \label{e:reflection_T}
\end{align}

\begin{thebibliography}{99}
\bibitem{laney_computational_1998} C. B. Laney, Computational Gasdynamics.
Cambridge: Cambridge University Press, 1998.
\end{thebibliography}

\end{document}
